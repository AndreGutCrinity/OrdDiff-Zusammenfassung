\section{Classification}
\noindent\rule[\linienAbstand]{\linewidth}{\linienDickeDick}

Differential equations can be classified according to various criteria. Besides the order of an ODE we are also interested in whether an ODE is linear, homogeneous, has constant coefficient, is separable or autonomous.\\

\textbf{Linearity}\\
An \emph{n-th} order ODE is \emph{linear}, if it is of the form:
\begin{equation}
  a_n(x) \cdot y^{(n)} + ... + a_1(x) \cdot y' + a_0(x) \cdot y = g(x)
\end{equation}
where $a_n(x), ..., a_1(x), a_0(x)$ are $g(x)$ fixed functions. Or in other words: A differential equation is linear if the dependant variable and all of its derivatives appear in a linear fashion (i.e., they are not multiplied together or squared for example or they are not part of transcendental functions such as sins, cosines, exponentials, etc.)\\\\

\textbf{Homogenity}\\
A lienar ODE is \emph{homogeneous}, if $g(x) = 0$ for all $x$; otherwise the ODE is \emph{inhomogeneous}, and $g(x)$ is the \emph{inhomogeneity} or \emph{source} term.\\

\textbf{Constant coefficient}\\
A linear ODE has \emph{constant coefficients}, if it is of the form
\begin{equation}
  a_n \cdot y^{(n)} + ... + a_1 \cdot y'+ a_0 \cdot y = g(x),
\end{equation}
with $a_n \neq 0$ (the source term $g(x)$ does not have to be constant).\\

\textbf{Separability}\\
The ODE is \emph{separable}, if $F(x, y)$ can be written as a product of a $x$- and $y$-dependent term, i.e. if the ODE is of the form
\begin{equation}
  y' = g(x) \cdot h(y)
\end{equation}

\textbf{Autonomity}\\
The ODE (1.28) is \emph{autonomous}, if $F(x, y)$ only depends on $y$, i.e. if the ODE is of the form
\begin{equation}
  y' = h(y)
\end{equation}
Every autonomous ODE is separable with $g(x) = 1$.\\

\textbf{Examples}\\
\begin{tabular}{ll}
  $y' = f(x)$ & Inhomogeneous linear ODE for $y(x)$ with\\
  & source term $f(x)$\\
  $m \cdot \dot{v} = m \cdot g - k \cdot v^2$ & Nonlinear ODE for $v(t)$\\
  $l \cdot \ddot{\Phi} + g \cdot sin(\Phi) = 0$ & Nonlinear ODE for $\Phi(t)$\\
  $l \cdot \ddot{\Phi} + g \cdot \phi$ & Homogeneous linear ODE for $\Phi(t)$\\
  $l \cdot \ddot{\Phi} + g \cdot \phi = sin(\omega t)$ & Inhomogeneous linear ODE for $\Phi(t)$ with\\
  & source term $sin(\omega t)$\\
  $i''+ \frac{R}{L}i' + \frac{1}{LC}i = 0$ & Homogeneous linear ODE for $i(t)$
\end{tabular}
