\section{Classification}
\noindent\rule[\linienAbstand]{\linewidth}{\linienDickeDick}

Differential equations can be classified according to various criteria. Besides the order of an ODE we are also interested in whether an ODE is linear, homogeneous, separable or autonomous.\\

\textbf{Linearity}\\
An \emph{n-th} order ODE is \emph{linear}, if it is of the form:
\begin{equation*}
  a_n(x) \cdot y^{(n)} + ... + a_1(x) \cdot y' + a_0(x) \cdot y = g(x)
\end{equation*}
where $a_n(x), ..., a_1(x), a_0(x)$ are $g(x)$ fixed functions.\\

\textbf{Homogenity}\\
A lienar ODE is \emph{homogeneous}, if $g(x) = 0$ for al $x$; otherwise the ODE is \emph{inhomogeneous}, and $g(x)$ is the \emph{inhomogeneity} or \emph{source} term.\\

\textbf{Constant coefficient}\\
A linear ODE has constant coefficients, if it is of the form
\begin{equation*}
  a_n \cdot y^{(n)} + ... + a_1 \cdot y'+ a_0 \cdot y = g(x),
\end{equation*}
with $a_n \neq 0$.\\

\textbf{Examples}\\
\begin{tabular}{ll}
  $y' = f(x)$ & Inhomogeneous linear ODE for $y(x)$ with\\
  & source term $f(x)$\\
  $m \cdot \dot{v} = m \cdot g - k \cdot v^2$ & Nonlinear ODE for $v(t)$\\
  $l \cdot \ddot{\Phi} + g \cdot sin(\Phi) = 0$ & Nonlinear ODE for $\Phi(t)$\\
  $l \cdot \ddot{\Phi} + g \cdot \phi$ & Homogeneous linear ODE for $\Phi(t)$\\
  $l \cdot \ddot{\Phi} + g \cdot \phi = sin(\omega t)$ & Inhomogeneous linear ODE for $\Phi(t)$ with\\
  & source term $sin(\omega t)$\\
  $i''+ \frac{R}{L}i' + \frac{1}{LC}i = 0$ & Homogeneous linear ODE for $i(t)$
\end{tabular}

\textbf{Separability}\\
