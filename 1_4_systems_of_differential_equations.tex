\section{Systems of differential equations}
\noindent\rule[\linienAbstand]{\linewidth}{\linienDickeDick}
If several systems are coupled with each other and mutually influence each other, one often obtains a system of ODE’s.\\
A \emph{system of differential equations} of first order is a system
\begin{equation}
  \begin{matrix}
    y'_1 & = & f_1(x, y_1,...,y_n)\\
    \vdots & & \vdots \;\;\;\;\; \; \; \; \; \; \; \; \; \; \; \; \; \; \\
    y'_n & = & f_n(x, y_1,...,y_n)
\end{matrix}
\end{equation}
of ODE’s for unknown functions $y_1(x), ... , y_n(x)$.\\
Using the vectorial notation
\begin{equation}
  \mathbf{y}' = \mathbf{f}(x, \mathbf{y})
\end{equation}

An ODE of \emph{n}-th order is equivalent to a system of first-order ODE's.
\begin{equation}
  \begin{matrix}
    y_1' & = & y_2\\
    y_2' & = & y_3\\
    \vdots  &  & \vdots \\
    y_n' & = & f(x, y_1, ..., y_n)
  \end{matrix}
\end{equation}

\textbf{Example: Higher-order ODE to system of first-order ODE's}\\
We want to rewrite the ODE
\begin{equation}
  y^4 + \textup{sin}(y'') \cdot e^{(x^2 + y)} = 0
\end{equation}
as a first order system. The solution is
\begin{equation}
  \begin{split}
    y_1' &= y_2\\
    y_2' &= y_3\\
    y_3' &= y_4 \\
    y_4' &= - \textup{sin}(y_3) \cdot e^{(x^2 + y_1)}
  \end{split}
\end{equation}
Where $y_1 = y,\; y_2 = y',\; y_3 = y'',\; y_4 = y'''$.\\
To make this system autonomous we introduce $y_0 = x$ and add $y_0' = 1$ to the system above.\\

\textbf{Example: 2nd-order ODE to system of first-order ODE's}\\
We want to rewrite the following 2$nd$ order ODE into a system of first-order ODE's.
\begin{equation}
  \ddot{x}(t) + 2\delta\dot{x}(t) + \omega_0^2x(t) = f(t)
\end{equation}
If we introduce the vector-valued function
\begin{equation}
  \textbf{y} = \begin{pmatrix} y_1 \\ y_2 \end{pmatrix} =
                \begin{pmatrix} x \\ \dot{x} \end{pmatrix} \;\;\;\;  \Rightarrow \;\;\;\;
                \dot{\textup{y}} =
                \begin{pmatrix} \dot{y_1} \\ \dot{y_2} \end{pmatrix} =
                \begin{pmatrix} \dot{x} \\ \ddot{x} \end{pmatrix}
\end{equation}
rewriting:
\begin{equation}
  \begin{split}
      \dot{\textbf{y}} =& \begin{pmatrix} \dot{x} \\ \ddot{x} \end{pmatrix} =
                          \begin{pmatrix} \dot{x} \\ -2\delta\dot{x}(t) - \omega_0^2x(t) + f(t) \end{pmatrix}\\
      \dot{\textbf{y}} =& \begin{pmatrix} \dot{x} \\ \ddot{x} \end{pmatrix} =
                          \begin{pmatrix} 0 & 1 \\ -\omega_0^2 & -2\delta \end{pmatrix}
                          \begin{pmatrix} x\\ \dot{x} \end{pmatrix} +
                          \begin{pmatrix} 0 \\ f(t) \end{pmatrix}\\
      \dot{\textbf{y}} =& \textup{A}\textbf{y} + \mathbf{b}
  \end{split}
\end{equation}

\textbf{Example: Coupled second order syste to first-oder ODE's}\\
Rewrite the second order system (coupled pendulums)
\begin{equation}
  \begin{split}
    \ddot{\varphi}_1(t) = -\beta \varphi_1 + \alpha \cdot (\varphi_2 - \varphi_1)\\
    \ddot{\varphi}_2(t) = -\beta \varphi_2 + \alpha \cdot (\varphi_1 - \varphi_2)
  \end{split}
\end{equation}
as a system of first order ODE.\\
With the new variables $\varphi_3 = \dot{\varphi}_1,\; \varphi_4 = \dot{\varphi}_2$ the system becomes
\begin{equation}
  \begin{split}
    \dot{\varphi}_1(t) &= \varphi_3\\
    \dot{\varphi}_2(t) &= \varphi_4\\
    \dot{\varphi}_3(t) &= -\beta \varphi_1 + \alpha \cdot (\varphi_2 - \varphi_1)\\
    \dot{\varphi}_4(t) &= -\beta \varphi_2 + \alpha \cdot (\varphi_1 - \varphi_2)
  \end{split}
\end{equation}
