\section{Analytical methods for first-order ODE’s}
\noindent\rule[\linienAbstand]{\linewidth}{\linienDickeDick}

\subsection{Overview}
\noindent\rule[\linienAbstand]{\linewidth}{\linienDicke}
For some types of explicit first-oder ODE’s there exist analytical solution methods:\\

\textbf{Separable ODE's}\\
We discuss this type of ODE’s in detail. The main reason why separable ODE’s are interesting is the fact that the special cases “Indefinite integral” and “Autonomous ODE” are of this type, and many important examples from applications are autonomous ODE’s.\\

\textbf{Linear ODE's}\\
We discuss linear ODE’s in the following section in the context of the discussion of linear ODE’s of arbitrary order. For linear ODE’s the distinction between homogeneous and inhomogeneous equations is crucial. A homogeneous first-order ODE is separable and can therefore be solved by the standard procedure for separable ODE’s. For inhomogeneous ODE’s, one often arrives at a solution by assuming an ansatz of the form of the inhomogeneity.\\


\subsection{Separable ODE’s}
\noindent\rule[\linienAbstand]{\linewidth}{\linienDicke}
How to find the general solution of a separable ODE is best expalined with an example:\\
We compute the general solution of the ODE
\begin{equation}
  y' = -\frac{x}{y}
\end{equation}
We write the eqution as
\begin{equation}
  \frac{\textup{d}y}{\textup{d}x} = -\frac{x}{y}
\end{equation}
We bring all $x$-dependent terms to the left hand side and all $y$-dependent terms on
the right hand side:
\begin{equation}
  y\:\textup{d}y = -x \:\textup{d}x
\end{equation}
We integrate on both sides and get
\begin{equation}
  \int y\:\textup{d}y = - \int x\:\textup{d}x \Rightarrow \frac{1}{2}y^2 = - \frac{1}{2} x^2 + C, \;\;\; C \in \mathbb{R}
\end{equation}
We solve for y and get
\begin{equation}
  y = \pm \sqrt{K - x^2}, \;\; K \in \mathbb{R}\;\; (\text{where } K = 2C)
\end{equation}

% \textbf{Example of substitution}\\
% We consider the ODE
% \begin{equation}
%   y' = (x + y)^2
% \end{equation}
%
%
% \subsection{Exact ODE’s}
% \noindent\rule[\linienAbstand]{\linewidth}{\linienDicke}
