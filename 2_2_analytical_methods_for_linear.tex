\section{Analytical methods for linear ODE’s}
\noindent\rule[\linienAbstand]{\linewidth}{\linienDickeDick}

\subsection{Overview}
\noindent\rule[\linienAbstand]{\linewidth}{\linienDicke}
We differentiate bewteen first-order linear ODE's and higher-oder ODE's as well as between homogeneous and inhomogeneous ODE's.\\

The general solution of the inhomogeneous ODE is the sum
\begin{equation}
  y = y_h + y_s,
\end{equation}
where $y_h$ is the general solution of the homogeneous ODE and $y_s$ any special solution of the inhomogeneous ODE.

\subsection{First-order linear ODE's}
\noindent\rule[\linienAbstand]{\linewidth}{\linienDicke}
To solve a first-order linear ODE we thus have to find $y_h$ and $y_s$\\
$\mathbf{y_h}$: a homogeneous first-order ODE is separable and can therefore be solved by the standard procedure for separable ODE’s described above.\\
$\mathbf{y_s}$: To find a special solution of and ODE, there are several possibilities. In the case of an ODE with constant coefficients, it usually suffices to choose for $y_s$ an \emph{ansatz of the form of the source term $g(x)$}. In the case of non-constant coefficients, the method \emph{variation of constants} usually works better.

\textbf{Example using the ansatz}\\
We solve the ODE
\begin{equation}
  y' + ay = b
\end{equation}
- To determine $y_h$, we integrate the homogeneous ODE
\begin{equation}
  y' + a_y = 0
\end{equation}
by separation of variables we obtain the solution
\begin{equation}
  y_h = C \cdot e^{-ax}, \;\; C \in \mathbb{R}.
\end{equation}
- For finding $y_s$ we choose the ansatz in the form of the source term. In this case the source term is constant, $g(x) = b$. Therefore we assume that the special solution $y_s$ is constant as well, i.e. we make an ansatz $y_s = c$. We plug this ansatz into the inhomogeneous ODE and obtain the special solution $y_s$.
\begin{equation}
  y_s'+ay_s = b \;\;\; \Rightarrow \;\;\; y_s = \frac{a}{b}
\end{equation}
The general solution therefore is
\begin{equation}
  y = C \cdot e^{-ax} + \frac{a}{b}, \;\; C \in \mathbb{R}
\end{equation}

\textbf{Ansatz functions for the solution of the inhomogeneous first-order ODE}
\begin{table}[H]
  \begin{tabular}{ll}
    Source term $g(x)$ & Ansatz $y_s$\\
    $g(x) = b_0$ & $y_s = c_0$\\
    $g(x) = b_1x + b_0$ & $y_s = c_1x + c_0$\\
    $g(x) = b_2x^2 + b_1x + b_0$ & $y_s = c_2 x^2 + c_1x + c_0$\\
    $g(x) = \sum_{i = 0}^n b_i x^i$ & $y_s = \sum_{i = 0}^n c_i x^i$\\
    $g(x) = A\;sin(\omega x) + B\;cos(\omega x)$ & $
    \begin{aligned}
      y_s &= C_1sin(\omega x) + C_2 cos(\omega x)\\
      y_s &= Csin(\omega x + \varphi)
    \end{aligned}$\\
    $g(x) = Ae^{bx}$ & $y_s =
    \left\{\begin{matrix}
      \frac{A}{b + a}e^{bx} \;\; \text{ for } b \neq -a\\
      Axe^{-ax} \;\text{ for } b = -a
    \end{matrix}\right.$
  \end{tabular}
\end{table}
In addition, the following rules must be followed:\\
\textbf{Linearity} If $g(x)$ is a linear combination of several source terms, one has to assume as ansatz for $y_s(x)$ a correponding linear combination of several ansatz terms.\\
\textbf{Resonance} If the source term $g(x)$ is itself already a solution of the homogeneous ODE, the correponding ansatz for $y_s$ has to be multiplied with x. So if for example $y_h = Ce^x$, and $g(x) = e^x$, the ansatz $y_s = \mathbf{x} \cdot e^x$ is choosen.

\textbf{Example using separation of variables}\\
