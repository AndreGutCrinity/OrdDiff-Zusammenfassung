\section{Analytical methods for linear ODE’s}
\noindent\rule[\linienAbstand]{\linewidth}{\linienDickeDick}

\subsection{Overview}
\noindent\rule[\linienAbstand]{\linewidth}{\linienDicke}
We differentiate bewteen first-order linear ODE's and higher-oder ODE's as well as between homogeneous and inhomogeneous ODE's.\\

The general solution of the inhomogeneous ODE is the sum
\begin{equation}
  y = y_h + y_s,
\end{equation}
where $y_h$ is the general solution of the homogeneous ODE and $y_s$ any special solution of the inhomogeneous ODE.

\subsection{First-oder linear ODE's}
\noindent\rule[\linienAbstand]{\linewidth}{\linienDicke}
The homogeneous ODE is separable and can therefore be integrated. One finds as general solution of the homogeneous ODE $y 0 + f(x) · y = 0$:
\begin{equation}
  y_h = K \cdot d^{-F(x)}, \;\; K \in \mathbb{R},
\end{equation}
where $F(x)$ is an andtiderivative of $f(x)$\\
