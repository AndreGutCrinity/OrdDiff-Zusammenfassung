\section{Analytical methods for linear systems}
\noindent\rule[\linienAbstand]{\linewidth}{\linienDickeDick}
By the definition given above, by a system of ODE’s we mean the following system of explicit firstoder ODE’s:
\begin{equation}
  \begin{matrix}
    \dot{x}_1 & = & f_1(t, x_1, ..., x_n)\\
    \vdots  &  & \vdots \\
    \dot{x}_n & = & f_n(t, x_1, ..., x_n)
  \end{matrix}
\end{equation}

\subsection{Overview}
\noindent\rule[\linienAbstand]{\linewidth}{\linienDicke}
A system of linear first-order ODE’s has the form
\begin{equation}
  \begin{matrix}
    \dot{x}_1 & = & a_{11}(t)x_1 + ... + a_{1n}(t)x_n + b_1 (t)\\
    \vdots  &  & \vdots \\
    \dot{x}_n & = & a_{n1}(t)x_1 + ... + a_{nn}(t)x_n + b_n (t)
  \end{matrix}
\end{equation}
or in matrix-vector notation
\begin{equation}
  \dot{\mathbf{x}} = A(t)\mathbf{x}+\mathbf{b}t
\end{equation}
The general solution of the inhomogeneous system is the sum
\begin{equation}
  \mathbf{x} = \mathbf{x}_h+ \mathbf{x}_s
\end{equation}
where $x_h$ is the general solution of the homogeneous system and $x_s$ any special solution of the inhomogeneous system.\\

The set of solutions of the homogeneous system is an n-dimensional vector space. This is equivalent to the following two statements:\\
 - Any linear combination of solutions is again a solution, i.e. if $x_1$ and $x_2$ are solutions, then $C_1x_1 + C_2x_2$ is also a solution for any $C_1, C_2 \in \mathbb{R}$\\
 - There exist precisely $n$ linearly independent solution $x_1, . . . , x_n$. Such a set $\{x_1, . . . , x_n\}$ of linearly independent solutions is also called a \emph{fundamental system of solutions}. Algebraically, a fundamental system thus is a basis of the vector space of solutions.\\

\subsection{Homogeneous linear systems}
\noindent\rule[\linienAbstand]{\linewidth}{\linienDicke}
Since in the scalar case the solution of a homogeneous linear ODE is given by $x = e^{at}$, we try in the vectorial case an ansatz of the form
\begin{equation}
  \mathbf{x} = e^{\lambda t} \cdot c =
  \begin{pmatrix}
    e^{\lambda t}c_1\\
    \vdots\\
    e^{\lambda t}c_n
  \end{pmatrix}
\end{equation}
Plugging the ansatz into the ODE leads to $e^{\lambda t}\lambda c = e^{\lambda t} Ac$ and we thus get for $\lambda$ and $c$ the equation
\begin{equation}
  A\mathbf{c} = \lambda \mathbf{c}
\end{equation}
This precisely means that $c$ is an \emph{eigenvector} of $A$ to the \emph{eigenvalue} $\lambda$.\\

\textbf{Example}\\
We compute the general solution of the homogeneous system
\begin{equation}
  \begin{split}
    \dot{x}_1 &= 2x_1 - x_2\\
    \dot{x}_2 &= -x_1 + 2x_2
  \end{split}
\end{equation}

or in the form $\dot{\mathbf{x}} = A\mathbf{x}$ with
\begin{equation}
  \mathbf{x}(t) = \begin{pmatrix}
    x_1(t)\\
    x_2(t)
  \end{pmatrix}
  , \;\; A = \begin{pmatrix}
      2 & -1\\
      -1 & 2
  \end{pmatrix}
\end{equation}

The matrix $A$ has the eigenvaluess $\lambda_1 = 1, \lambda_2 = 3$ with corresponding eigenvectors $v_1 = \binom{1}{1}, v_2 = \binom{1}{-1}$. Hence the general solution $\mathbf{x}(t)$ of the homogeneous system is
\begin{equation}
  \mathbf{x}(t) = \begin{pmatrix}
    x_1(t)\\
    x_2(t)
  \end{pmatrix} = C_1 \begin{pmatrix} 1 \\ 1 \end{pmatrix} e^t+
  C_2 \begin{pmatrix} 1 \\ -1 \end{pmatrix} 3^{3t}
\end{equation}

\subsection{Inhomogeneous linear systems}
\noindent\rule[\linienAbstand]{\linewidth}{\linienDicke}
