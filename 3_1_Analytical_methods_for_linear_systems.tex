\section{Analytical methods for linear systems}
\noindent\rule[\linienAbstand]{\linewidth}{\linienDickeDick}
By the definition given above, by a system of ODE’s we mean the following system of explicit firstoder ODE’s:
\begin{equation}
  \begin{matrix}
    \dot{x}_1 & = & f_1(t, x_1, ..., x_n)\\
    \vdots  &  & \vdots \\
    \dot{x}_n & = & f_n(t, x_1, ..., x_n)
  \end{matrix}
\end{equation}

\subsection{Overview}
\noindent\rule[\linienAbstand]{\linewidth}{\linienDicke}
A system of linear first-order ODE’s has the form
\begin{equation}
  \begin{matrix}
    \dot{x}_1 & = & a_{11}(t)x_1 + ... + a_{1n}(t)x_n + b_1 (t)\\
    \vdots  &  & \vdots \\
    \dot{x}_n & = & a_{n1}(t)x_1 + ... + a_{nn}(t)x_n + b_n (t)
  \end{matrix}
\end{equation}
or in matrix-vector notation
\begin{equation}
  \dot{\mathbf{x}} = A(t)\mathbf{x}+\mathbf{b}t
\end{equation}
The general solution of the inhomogeneous system is the sum
\begin{equation}
  \mathbf{x} = \mathbf{x}_h+ \mathbf{x}_s
\end{equation}
where $x_h$ is the general solution of the homogeneous system and $x_s$ any special solution of the inhomogeneous system.\\

The set of solutions of the homogeneous system is an n-dimensional vector space. This is equivalent to the following two statements:\\
 - Any linear combination of solutions is again a solution, i.e. if $x_1$ and $x_2$ are solutions, then $C_1x_1 + C_2x_2$ is also a solution for any $C_1, C_2 \in \mathbb{R}$\\
 - There exist precisely $n$ linearly independent solution $x_1, . . . , x_n$. Such a set $\{x_1, . . . , x_n\}$ of linearly independent solutions is also called a \emph{fundamental system of solutions}. Algebraically, a fundamental system thus is a basis of the vector space of solutions.\\

\subsection{Homogeneous linear systems}
\noindent\rule[\linienAbstand]{\linewidth}{\linienDicke}
Since in the scalar case the solution of a homogeneous linear ODE is given by $x = e^{at}$, we try in the vectorial case an ansatz of the form
\begin{equation}
  \mathbf{x} = e^{\lambda t} \cdot \mathbf{c} =
  \begin{pmatrix}
    e^{\lambda t}c_1\\
    \vdots\\
    e^{\lambda t}c_n
  \end{pmatrix}
\end{equation}
Plugging the ansatz into the ODE leads to $e^{\lambda t}\lambda \mathbf{c} = e^{\lambda t} A\mathbf{c}$ and we thus get for $\lambda$ and $\mathbf{c}$ the equation
\begin{equation}
  A\mathbf{c} = \lambda \mathbf{c}
\end{equation}
This precisely means that $\mathbf{c}$ is an \emph{eigenvector} of $A$ to the \emph{eigenvalue} $\lambda$.
We destinguish between the following cases for $\lambda$\\

\textbf{Real eigenvalues}\\
If all eigenvalues of $A$ are real and $A$ has $n$ linear independent eigenvectors then the general solution is a linear combination of
\begin{equation}
  \mathbf{x}(t) = e^{\lambda t} \cdot \mathbf{c}
\end{equation}

Example: We compute the general solution of the homogeneous system
\begin{equation}
  \begin{split}
    \dot{x}_1 &= 2x_1 - x_2\\
    \dot{x}_2 &= -x_1 + 2x_2
  \end{split}
\end{equation}

or in the form $\dot{\mathbf{x}} = A\mathbf{x}$ with
\begin{equation}
  \mathbf{x}(t) = \begin{pmatrix}
    x_1(t)\\
    x_2(t)
  \end{pmatrix}
  , \;\; A = \begin{pmatrix}
      2 & -1\\
      -1 & 2
  \end{pmatrix}
\end{equation}

The matrix $A$ has the eigenvalues $\lambda_1 = 1, \lambda_2 = 3$ with corresponding eigenvectors $v_1 = \binom{1}{1}, v_2 = \binom{1}{-1}$. Hence the general solution $\mathbf{x}(t)$ of the homogeneous system is
\begin{equation}
  \mathbf{x}(t) = \begin{pmatrix}
    x_1(t)\\
    x_2(t)
  \end{pmatrix} = C_1 e^t \begin{pmatrix} 1 \\ 1 \end{pmatrix} +
  C_2 e^{3t} \begin{pmatrix} 1 \\ -1 \end{pmatrix}
\end{equation}

\textbf{Complex eigenvalues}\\
If there is at least one pair of complex conjugate eigenvalues and $A$ has $n$ linear independent eigenvectors then the general (complex) solution is a linear combination of $\mathbf{x} = e^{\lambda t}\cdot \mathbf{c}$. Real and complex parts of complex solutions are real
solutions


If $\lambda = \mu + j\nu$ is a simple complex eigenvalue of $A$ and $\mathbf{c} = \mathbf{a} + j\mathbf{b}$ a corresponding complex eigenvector of $A$, then from the complex solution
\begin{equation}
  \mathbf{x}(t) = e^{\lambda t} \mathbf{c} = e^{(\mu + j\nu)t} (\mathbf{a} + j\mathbf{b})
\end{equation}
we get two linearly independent real solutions by separating the complex solution into real and imaginary parts:
\begin{equation}
  \begin{split}
    \mathbf{z}_1(t) &= \textup{Re}(e^{\lambda t} \mathbf{c}) = e^{\mu t}(\mathbf{a} \; cos(\nu t) - \mathbf{b} \; sin(\nu t))\\
    \mathbf{z}_2(t) &= \textup{Im}(e^{\lambda t} \mathbf{c}) = e^{\mu t}(\mathbf{a} \; sin(\nu t) + \mathbf{b} \; cos(\nu t))
  \end{split}
\end{equation}
The complex conjugate of $\lambda$, namely $c = a - jb$ does not have to be considered because it would again lead to the same solutions.\\
x is then z1 + z2 ???\\

Example: We compute the general solution of a homogeneous system of ODE's where the matrix $A$ is
\begin{equation}
  A =
  \begin{pmatrix}
    3 & -2\\
    4 & -1
  \end{pmatrix}
\end{equation}
The eigenvalues of $A$ are $\lambda_{1, 2} = 1 \pm 2j$, with eigenvectors $\lambda_{1,2} = \begin{pmatrix}\color{red} 1 \pm j \\ \color{red}2 \color{black}\end{pmatrix}$.
\begin{equation}
  \mathbf{x} = C_1e^t \begin{pmatrix}
  \color{red}1\color{black}\textup{cos}(2t) - \color{red}1\color{black}\textup{sin}(2t)\\
  \color{red}2\color{black}\textup{cos}(2t) - \color{lightgray}0\textup{sin}(2t)\color{black} \end{pmatrix} +
  C_2 e^t \begin{pmatrix}
  \color{red}1\color{black}\textup{sin}(2t) +
  \color{red}1\color{black}\textup{cos}(2t)\\
  \color{red}2\color{black}\textup{sin}(2t) +
  \color{lightgray}0\textup{cos}(2t)\color{black}\end{pmatrix}
\end{equation}

\textbf{Polynomial expressions}\\
If $A$ has multiple eigenvalues and less than $n$ linear independent eigenvectors (More eigenvalues than eigenvectors), then not all solutions are of the form $\mathbf{x}(t) = e^{\lambda t} \mathbf{c}$. Additional solutions involve polynomial expressions. These polynomials are of the form\\
\begin{equation}
  \begin{split}
    p_0(t) &= e^{\lambda t} \mathbf{v}_1\\
    p_1(t) &= e^{\lambda t} (t\mathbf{v}_1 + \mathbf{v}_2)\\
    p_2(t) &= e^{\lambda t} (t^2\mathbf{v}_1 + 2t\mathbf{v}_2 + 2\mathbf{v}_3)
  \end{split}
\end{equation}

Example: The only eigenvalue of the matrix $A = \begin{pmatrix} 2 & 1 \\ 0 & 2 \end{pmatrix}$ is $\lambda = 2$, with the single (linearly
independent) eigenvector $\mathbf{v} = \binom{1}{0}$. Let $\mathbf{v}_1 = \mathbf{v}$\\
To find a generalized eigenvector $\mathbf{v}_2$, we solve the equation
\begin{equation}
  (A - \lambda E_n)\mathbf{v}_2 = \mathbf{v}_1
\end{equation}
\begin{equation}
  \begin{split}
    \left(\begin{pmatrix} 2 & 1 \\ 0 & 2 \end{pmatrix} -
    \begin{pmatrix} 2 & 0 \\ 0 & 2 \end{pmatrix} \right) \mathbf{v}_2 =&
    \begin{pmatrix} 1 \\ 0 \end{pmatrix}\\
    \begin{pmatrix} 0 & 1 \\ 0 & 0 \end{pmatrix} \mathbf{v}_2 =&
    \begin{pmatrix} 1 \\ 0 \end{pmatrix}\\
    \begin{pmatrix} 0 & 1 \\ 0 & 0 \end{pmatrix} \begin{pmatrix} 0 \\ 1 \end{pmatrix} =&
    \begin{pmatrix} 1 \\ 0 \end{pmatrix}
  \end{split}
\end{equation}
And therefore
\begin{equation}
  \mathbf{p}_1(t) = t\mathbf{v}_1 + \mathbf{v}_2 = t
  \begin{pmatrix} 1 \\ 0\end{pmatrix} +
  \begin{pmatrix} 0 \\ 1  \end{pmatrix} =
  \begin{pmatrix} t \\ 1 \end{pmatrix}
\end{equation}

The general solution therefore is
\begin{equation}
  \begin{split}
    \mathbf{x}(t) =& C_1 e^{2t} \begin{pmatrix} 1 \\ 0\end{pmatrix} +
    C_2 e^{2t} \begin{pmatrix} t \\ 1 \end{pmatrix}\\
    \mathbf{x}(t) =& \begin{pmatrix} C_1 e^{2t} + C_2 e^{2t}\\
    C_2 e^{2t} \end{pmatrix}
  \end{split}
\end{equation}



\subsection{Inhomogeneous linear systems}
\noindent\rule[\linienAbstand]{\linewidth}{\linienDicke}
We now discuss the inhomogeneous system
\begin{equation}
  \dot{\mathbf{x}} = A\mathbf{x} + \mathbf{b}t
\end{equation}
with a constant coefficient matrix $A$ and a nonconstant source term $\mathbf{b}(t)$.\\
We will look at the following two methods for obtaining the special solution of the homogeneous system.
\begin{itemize}
  \item Elimination of variables
  \item Using the decomposition $\mathbf{x} = \mathbf{x}_h + \mathbf{x}_s$ and choosing an ansatz for $\mathbf{x}_s$ of a similar type as the source term.
\end{itemize}

\textbf{Elemination of variables}\\
We solve the IVP
\begin{equation}
  \begin{split}
    \dot{x}_1 &= 2x_1 - x_2 + 1, \;\;\; x_1(0) = 0\\
    \dot{x}_2 &= -x_1 + 2x_2 - 1, \;\;\; x_2(0) = 1
  \end{split}
\end{equation}

\begin{equation}
  x_2 = -\dot{x}_1 + 2x_1 + 1
\end{equation}

\begin{equation}
  \ddot{x}_1 -4\dot{x}_1 + 3x_1 + 1 = 0
\end{equation}

\begin{equation}
  x_1 = C_1 e^t + C_2 e^{3t} - \frac{1}{3}
\end{equation}

\begin{equation}
  x_2 = C_1 e^t - C_2 e^{3t} + \frac{1}{3}
\end{equation}



\begin{equation}
  \begin{vmatrix}
    C_1 e^t + C_2 - \frac{1}{3} = 0\\
    C_1 - C_2 + \frac{1}{3} = 1
  \end{vmatrix}
\end{equation}
for $C_1$ and $C_2$. The solution of this system is
\begin{equation}
  C_1 = \frac{1}{2}, \;\;\;\;\; C_2 = -\frac{1}{6}
\end{equation}
The IVP thus has the unique solution
\begin{equation}
  \begin{pmatrix} x_1 \\ x_2 \end{pmatrix} =
  \begin{pmatrix}
    \frac{1}{2}e^t - \frac{1}{6}e^{3t} - \frac{1}{3}\\
    \frac{1}{2}e^t + \frac{1}{6}e^{3t} + \frac{1}{3}
  \end{pmatrix}
\end{equation}


\textbf{Choice of an ansatz}\\
\begin{equation}
  \begin{pmatrix} x_1 \\ x_2 \end{pmatrix} =
  \begin{pmatrix}
    \frac{1}{2}e^t - \frac{1}{6}e^{3t} - \frac{1}{3}\\
    \frac{1}{2}e^t + \frac{1}{6}e^{3t} + \frac{1}{3}
  \end{pmatrix}
\end{equation}
