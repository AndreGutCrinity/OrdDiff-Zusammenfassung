\section{Overview}
\noindent\rule[\linienAbstand]{\linewidth}{\linienDickeDick}
A \emph{dynamical system} is a time-dependent process which is described by a mathematical model and whose temporal evolution is completely determined by its initial state.\\
The set of all possible states of a dynamical system is its \emph{phase space}, and the temporal evolution in phase space is the \emph{flow} of the dynamical system.
\begin{itemize}
  \item A \emph{continuous} dynamical system is a system of differential equations of the form
  \begin{equation}
    \dot{\mathbf{x}} = \mathbf{f}(\mathbf{x}), \;\; \mathbf{x} \in \mathbb{R}
  \end{equation}
  Time in such systems is in principle measured continuously.\\
  The solutions of these systems are differentiable functions. Solutions are also called \emph{trajectories} or \emph{orbits} of the system.

  \item A \emph{discrete} dynamical system is a system of difference equations of the form
  \begin{equation}
    x_{n + 1} = f(x_n)
  \end{equation}
  Time in physical systems sometimes also has to measured in discrete steps.\\
  The solutions of this equation are sequences $(x_n)$.
\end{itemize}
