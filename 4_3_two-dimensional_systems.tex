\section{Two-Dimensional Systems}
\noindent\rule[\linienAbstand]{\linewidth}{\linienDickeDick}
We now consider the systmes for arbitrary $n \in \mathbb(N)$, i.e.
\begin{equation}
  \begin{split}
    \dot{x}_1 &= f_1(x_1, x_2,...,x_n)\\
    \dot{x}_2 &= f_2(x_1, x_2,...,x_n)\\
     &\;\;...\\
    \dot{x}_n &= f_n(x_1, x_2,...,x_n)
  \end{split}
\end{equation}
resp. in vector notation
\begin{equation}
  \dot{\mathbf{x}} = \mathbf{f}(\mathbf{x}) \;\;\; (\mathbf{x}\in\mathbb(R)^n)
\end{equation}



\textbf{Definitions}\\
A point $x^* \in \mathbb{R}^n$ is a \emph{fixed point} of the dynamical system $\dot{\mathbf{x}} =  \mathbf{f}(\mathbf{x})$, if
\begin{equation}
  \mathbf{f}(\mathbf{x^*}) = \mathbf{0}
\end{equation}
If $\mathbf{x^*}$ is a \emph{fixed point} of the dynamical system $\dot{\mathbf{x}} =  \mathbf{f}(\mathbf{x})$, then
\begin{equation}
   \mathbf{x}(t) =  \mathbf{x}^*
\end{equation}
is a constant solution.\\

A fixed point $x^* \in \mathbb{R}^n$ of the dynamical system $\dot{\mathbf{x}} =  \mathbf{f}(\mathbf{x})$, is
\begin{itemize}
  \item \emph{stable}, if the solution starting close to $x^*$ stays close to $x^*$ for all $t \geq 0$.
  \item \emph{asymptotically stable}, if the solution starting close to $x^*$ converge to $x^*$ for $t \rightarrow \infty$.
  \item \emph{unstable}, if the solution starts close to $x^*$ and diverging from $x^*$ for $t \rightarrow \infty$.
\end{itemize}

\subsection{Linear Systems}
\noindent\rule[\linienAbstand]{\linewidth}{\linienDicke}

Overview over the various types of fixed points and the corresponding geometry of the system near the fixed point:
\begin{figure}[H]
  \centering
  \includegraphics[width=.7\linewidth]{Pics/4.18.png}
\end{figure}

\begin{table}[H]
  \footnotesize
  \begin{tabular}{ll}
    a) saddle point    & e) unstable spiral\\
    b) stable node     & f) center/elliptic fixed point\\
    c) unstable node   & g) degenerate cases\\
    d) stable spiral   &
  \end{tabular}
\end{table}

A two-dimensional system can be characterized with the eigenvalues like so:
\begin{itemize}
  \item $\lambda_1 < 0, \lambda_2 < 0$: stable node
  \item $\lambda_1 > 0, \lambda_2 < 0$: saddle point
  \item $\lambda_1 > 0, \lambda_2 > 0$: usntable point
  \item $\lambda_1 = 0, \lambda_2 = 0$: non-isolated fixed point
  \item $\lambda_{1.2} \neq \mathbb{R}, \textup{Re}(\lambda_{1,2}) = 0$: center
  \item $\lambda_{1.2} \neq \mathbb{R}, \textup{Re}(\lambda_{1,2}) > 0$: unstable spiral
  \item $\lambda_{1.2} \neq \mathbb{R}, \textup{Re}(\lambda_{1,2}) < 0$: stable spiral
\end{itemize}

Alternativ characterization instead of with the eigenvalues $\lambda_1, \lambda_2$ with trace and determinant of A:
\begin{equation}
  \begin{split}
    \tau &= \textup{tr} = a_{11} + a_{22}\\
    \Delta &= \textup{det}(A) = a_{11}a_{22} - a_{12}a_{21}
  \end{split}
\end{equation}
Connection with the eigenvalues:
\begin{equation}
  \lambda_{1,2} = \frac{1}{2}\left(\tau \pm \sqrt{\tau^2 - 4\Delta}\right)
\end{equation}
and
\begin{equation}
  \Delta = \lambda_1 \lambda_2, \;\;\; \tau = \lambda_1 + \lambda_2
\end{equation}

In the following figure the classification of fixed points based on the properties of $\tau$ and $\Delta$ is shown graphically.
\begin{figure}[H]
  \centering
  \includegraphics[width=.7\linewidth]{Pics/4.19.png}
\end{figure}

If one is only interested in the stability of $\textbf{x}^*$ (and not in a further classification into nodes,
spirals, etc.), the fixed point $\textbf{x}^* = 0$ of a dynamical system is
\begin{itemize}
  \item asymptotically stable, if
  \begin{equation}
    \textup{Re}(\lambda_i) < 0
  \end{equation}
  holds for all eigenvalues $\lambda_i$ of $A$

  \item stable, if
  \begin{equation}
    \textup{Re}(\lambda_i) \leq 0
  \end{equation}
  holds for all eigenvalues $\lambda_i$ of $A$

  \item usntable, if
  \begin{equation}
    \textup{Re}(\lambda_i) > 0
  \end{equation}
  holds for at least one eigenvalue $\lambda_i$ of $A$
\end{itemize}

\subsection{Nonlinear Systems}
\noindent\rule[\linienAbstand]{\linewidth}{\linienDicke}
In order to use the analysis of linear systems, we linearize the nonlinear system near the fixed point $x^*$. To this end, we consider small perturbations $u = x_1 - x^*_1, \; v = x_2 - x^*_2$ and deduce equations for $u$ and $v$ by expanding $f_1$ and $f_2$ into Taylor series. The temporal evolution of $(u, v)$ is thus given by
\begin{equation}
  \begin{pmatrix} \dot{u} \\ \dot{v} \end{pmatrix} =
  \begin{pmatrix} \frac{\partial f_1}{\partial x_1} & \frac{\partial f_1}{\partial x_2}\\
                  \frac{\partial f_2}{\partial x_1} & \frac{\partial f_2}{\partial x_1}
  \end{pmatrix}_{(x_1^*, x_2^*)} \cdot
  \begin{pmatrix} u \\ v \end{pmatrix} + \text{higher-order terms}
\end{equation}
The matrix
\begin{equation}
  A =   \begin{pmatrix} \frac{\partial f_1}{\partial x_1} & \frac{\partial f_1}{\partial x_2}\\
                        \frac{\partial f_2}{\partial x_1} & \frac{\partial f_2}{\partial x_1}
        \end{pmatrix}_{(x_1^*, x_2^*)}
\end{equation}
is the Jacobi matrix of the system near the fixed point $\textbf{x}^* = (x_1^*, x_2^*)$.\\

Example: Consider the system
\begin{equation}
  \begin{split}
    \dot{x} &= -x + x^3\\
    \dot{y} &= -2y
  \end{split}
\end{equation}
fixed points are:
\begin{equation}
  P_1 = (0, 0), \;\; P_2 = (1, 0), \;\; P_3 = (-1, 0).
\end{equation}
The Jacobi matrix of the system is
\begin{equation}
  A = \begin{pmatrix}
      -1+3x^2 & 0 \\
      0 & -2
  \end{pmatrix}
\end{equation}
The eigenvalues of $A$ are:
\begin{equation}
  P_1: \lambda_1 = -1, \lambda_2 = -2; \;\;\;\;\; P_2, P_3: \lambda_{1,2} = \pm 2
\end{equation}
Therefore $P_1$ is a stable node, and $P_2$ and $P_3$ are saddle points of the linearized system.\\

Intuitively: The linearization leads to the correct classification of the fixed points of nonlinear systems, if the fixed point type is described by an inequality and not an equality.\\

\textbf{Hartman-Grobman theorem}\\
If both eigenvalues $\lambda_1, \lambda_2$ of the corresponding system matrix $A$ fulfill the condition $\textup{Re}(\lambda_i) \neq 0$, then the behaviour of the nonlinear system is for sufficiently small perturbations topologically equivalent to the behaviour of the linearized system.\\

\textbf{Classification of fixed points into robust and marginal cases}\\
\emph{Robust}: Nodes, spirals, saddle points\\
\emph{Marginal}: centers/elliptic fixed points, degenerate cases

\subsection{Limit cycles}
\noindent\rule[\linienAbstand]{\linewidth}{\linienDicke}
If a fixed point is of the type ``center'', it is surrounded by nothing but closed trajectories. If there is a single isolated trajectory in the system, one speaks of a limit cycle.\\
A limit cycle of a dynamical system is an isolated closed trajectory of the system.
\begin{table}[H]
  \setlength{\tabcolsep}{0.2em}
  \footnotesize
  \begin{tabular}{p{\linewidth / 2 - 0.5em}@{\hskip 1em}p{\linewidth / 2 - 0.5em}}
    \includegraphics[width=\linewidth]{Pics/4.16.png} &
    \includegraphics[width=\linewidth]{Pics/4.34.png}\\
    Non-isolated closed orbits & Isolated closed orbits, limit cycle
  \end{tabular}
\end{table}
Limit cycles are important structuring elements of the phase space of a 2D dynamical system, in addition to fixed points\\

\textbf{Methods for detection (Poincaré-Bendixson theorem)}\\
Let $\dot{\mathbf{x}} = f(x)$ be a dynamical system in a bounded and closed subset $G$ of $\mathbb{R}^2$. If the system does not have a fixed point in $G$ and if there exists a trajectory $C$, which stays in $G$ for all $t \geq t_0$, then $C$ is either a closed trajectory, or it converges to a closed trajectory.\\

If a trajectory stay confined to a bounded closed region without fixed point, it has to converge to a closed trajectory, if it is not already a closed trajectory itself. In particular, a chaotic behaviour is impossible! Note that the result holds only for $n = 2$.\\

Example: Consider the dynamical system:
\begin{equation}
  \begin{split}
    \dot{x} &= x + y - x(x^2 + y^2)\\
    \dot{y} &= -x + y -y(x^2 + y^2)
  \end{split}
\end{equation}

In order to apply the \emph{Poincaré-Bendixson theorem} we choose
\begin{equation}
  G = \left{(x, y), \in \mathbb{R}^2|0.5 \leq \sqrt{x^2 + y^2}\leq2\right}.
\end{equation}
i.e. a torus with inner radius of 0.5 and an outer radius of 2. The system has no fixed point in $G$ since the fixed point lays at the origin, which is not contained in $G$.\\
If the equations are formulated in polar coordinates one gets
\begin{equation}
  \begin{split}
    \dot{r} &= r(1-r^2)\\
    \dot{\theta} &= -1
  \end{split}
\end{equation}

For $r = 0.5$ we have $\dot{r}(t) > 0$, whereas for $r = 2$ have the reverse estimate $\dot{r}(t) < 0$. Therefore $r(t)$ \emph{increases} for a trajectory of the system at the inner limit of the torus $G$, whereas $r(t)$ \emph{decreases} for a trajectory at the outer limit of the region. This implies that a trajectory starting in $G$ must stay in $G$ for all times. The conditions for applying the \emph{Poincaré-Bendixson theorem} are thus fulfilled, and it follows from the theorem that the system has a limit cycle.

\subsection{Bifurications}
\noindent\rule[\linienAbstand]{\linewidth}{\linienDicke}
A bifurcation occurs in the system $\dot{x} = f(x, r)$, if the phase portrait changes its topological structure as the parameter $r$ is varied.\\

\begin{itemize}
  \item Importance of bifurcations: By knowing the bifurcations of the system under considerations, on can produce/avoid desired/undesired properties of the system by varying the parameters of the system.
  \item All bifurcations occuring in 1D systems also occur in 2D systems
  \item New type of bifurcation which occurs only in systems of dimension $n \leq 2$: Hopf bifurcation: Creation/deletion of a limit cycle by varying a system parameter. This describes the ways in which oscillations in a system can be turned on or of.
\end{itemize}

\textbf{Hopf Bifurication}\\
If in a linear stability analysis $\Delta > 0$ (the fixed point is a center) it is possible for a Hopf bifurcation to occur.
\begin{itemize}
  \item if $\tau < 0$, the fixed point is stable and thus, no limit cycle can occur.
  \item if $\tau > 0$, the fixed point is unstable and thus, limit cycles can occur. This has to be hecked with the \emph{Poincaré-Bendixson theorem}.
\end{itemize}
