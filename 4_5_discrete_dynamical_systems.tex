\section{Discrete Dynamical Systems}
\noindent\rule[\linienAbstand]{\linewidth}{\linienDickeDick}

\subsection{Lyapunov exponent}
\noindent\rule[\linienAbstand]{\linewidth}{\linienDicke}
The Lyapunov exponent $\lambda$ of the discrete dynamical system $x_{n+1} = f(x_n)$ for an orbit starting at $x_0$ is
\begin{equation}
  \lambda = \lim_{n \to \infty}\left(\frac{1}{n} \sum_{i=0}^{n-1} \textup{ln}|f'(x_i)|\right)
\end{equation}

Significance of the sign of $\lambda$:
\begin{itemize}
  \item $\lambda > 0$: The system behaves chaotically
  \item $\lambda < 0$: The system does not behave chaotically
  \item $\lambda = 0$: The system is at a bifurcation point
\end{itemize}

\subsection{Mandelbrot set}
\noindent\rule[\linienAbstand]{\linewidth}{\linienDicke}
The Julia set $J$ of the complex dynamical system $z_{n+1} = f(z_n)$ is the boundary between the set of starting points with bounded orbits and the set of starting points with unbounded orbits.


\subsection{Mandelbrot set}
\noindent\rule[\linienAbstand]{\linewidth}{\linienDicke}
The Mandelbrot set $M$ of the complex discrete dynamical system
\begin{equation}
  z_{n + 1} = z_n^2 + c
\end{equation}
is the set of parameters $c \in C$, for which the set $(z_n)_n \in N$ of points generated by the dynamics with initial value $z_0 = 0$ remains bounded:
\begin{equation}
  M = \left\{c\in C| f_c^n(0) \not\to \infty, \text{for}\; n \to \infty \right\}
\end{equation}
Remark. The Mandelbrot set thus is a set of parameters, for which the equation above has a certain properties, namely that the sequence of points starting at the origin and evolving by the dynamics remains bounded.
